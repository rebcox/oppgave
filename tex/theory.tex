\section{Theory}

\subsection{Haemodynamic response} \label{haemodynamic}
This section...
\subsubsection{What is}
\subsubsection{Hemoglobin}



\subsection{Principles of Diffuse Optical Spectroscopy}
This section...



\subsubsection{Optical Properties of Biological Tissue}
When light is sent by a source through the sculp, it will be scattered and to some degree absorbed before it reaches an optical detector. (cite: optical methods book) The light in the near-infrared range between roughly 600 and 950 nm is relaticely poorly absorbed by biological tissue. The primary absorbers in this wavelength range are oxyhemoglobin (HbO) and deoxyhemoglobin (HbR) (cite:noninvasive imaging of cerebral activation...). These are both important biological indicators (section \ref{haemodynamic}). The absorption spectra, in the near-infrared window, of HbO and HbR, as well as six other biological components, can be read from figure ... in Appendix ... (Fig. 2.3 i optical methods). 

\subsubsection{Beer-Lambert Law}

\subsubsection{The Modified Beer-Lambert Law}


\subsection{Optical methods}
This section...
\subsubsection{NIRS}
The amount of light absorbed is used to measure hemoglobin levels in the blood. 
\subsubsection{In this experiment}


\subsection{Motion artifact}
This section...
\subsubsection{Why is it a problem}
\subsubsection{Motion correction methods}